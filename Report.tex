\documentclass[12pt]{article}
\usepackage[a4 paper, portrait, margin=0.75in]{geometry}
\title{EGYPT\\Farmers to Pharoahs}
\author{Derrick Diana\\Harry Heathcock\\Kaedon Jon Williams}
\date{\today}
\usepackage{multicol}
\usepackage{graphicx}
\usepackage{indentfirst}
\graphicspath{ {Figures/} }

\begin{document}
	\maketitle
	\begin{abstract}
			
	\end{abstract}
	
	\section{Introduction}
		\subsection{Changes}
			A number of changes in the implementation were made, some due to what seemed to be implementation errors in the original code and others due to functionality not making sense with what was trying to be achieved.
			\subsubsection{Death Due to Lack of Grain}
				In the original NetLogo code, if a household did not have enough grain to survive a year then a single person would die and the grain would be set to zero. In reality this doesn't make sense, as if a household has no food, then no one would survive. The reimplementation changed this so that only the family members who can be fed will survive.\\
			\subsubsection{Population Increases}
				In the original code, if the total current population is less than or equal to the historical projected population then the population can increase. At first glance this seems to be correct, however this allows the population to always grow slightly faster than the projected historical value, and thus in the reimplementation this is changed so that the population will only increase if it is less than the historical projected population.
			\subsubsection{Negative Grain Production}

			\subsubsection{Distance Cost For Claiming Fields}

			\subsubsection{Seeding Cost For Claiming Fields}
				
			\subsubsection{Generational Changes in Ambition and Competency}
				In the original code, if the ambition or competency for a household moved out of the bounds set (minimum, up to 1) then a new value would be generated until it fell within the bounds. This means that a household which is very ambitious is more likely to become less ambitious than it is to stay at a high ambition (and vice versa for low values). In the reimplementation, if one of the values is out of the bounds, then the value is set to the bounding value.
			
	\section{Test Cases}

	\section{Conclusions}
	
	\begin{thebibliography}{9}

	\end{thebibliography}
\end{document}